\documentclass[12pt]{article}
\usepackage{setspace}
\usepackage{geometry}
\geometry{margin=1in}
\setstretch{1.3}

\begin{document}

\title{Final Year Project Management System}
\author{Sadia Khan}
\date{}
\maketitle

\section*{Introduction}
In our university, managing final year projects is not very easy. Students have to think of project ideas, submit them to their teachers, and then wait for approval. After approval, they have to keep updating their progress. Most of the time, this is done using paper forms or messages, which takes a lot of time and sometimes causes confusion. Important documents may get lost or students forget to submit updates on time.

Teachers also find it difficult to track the progress of all students at the same time. They have to check every student’s work manually, give feedback, and keep records, which is a lot of work. Admin staff also face problems when assigning projects or keeping all the information organized.

To solve these problems, we are developing a \textbf{Final Year Project Management System}. This system will allow students to submit project proposals online, get feedback from their supervisors, and update their progress easily. Supervisors can review all projects in one place, provide feedback quickly, and track deadlines. Admins can also manage users, assign projects, and generate reports without any hassle.

With this system, managing final year projects will become much faster, easier, and more organized. Students will know the status of their projects at any time, teachers can keep track of progress easily, and admin staff can handle everything without confusion. This system will make the whole process of final year projects smooth and efficient for everyone involved and will also save a lot of time and effort.

\section*{Problem Statement}
Managing final year projects in our university has a lot of problems right now. Students and teachers face many difficulties. First, students often don’t know how to submit their proposals properly, and sometimes they submit late or lose important documents. Teachers have to check every student project one by one, which takes too much time, and it’s hard to give feedback to everyone on time.

Communication is another problem. Students and supervisors often send messages or emails, but some messages get missed or delayed. This causes confusion and delays in project progress. Tracking progress is also hard because there is no system to see which projects are on time, which are delayed, or which students need help.

Admin staff face problems too. They have to assign supervisors to students manually and manage all the records. This takes a lot of effort and sometimes mistakes happen. All these problems make the process slow, confusing, and stressful for students, teachers, and admins, and it becomes difficult to manage final year projects properly.

\end{document}
